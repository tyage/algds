\documentclass[a4paper,12pt]{article}
\usepackage{listings}
\title{アルゴリズムとデータ構造入門 必修課題2}
\author{1029-24-9540 山崎啓太郎}
\begin{document}
\maketitle

\section{painterを1種類作成}
\lstset{numbers=left,basicstyle=\small}
\lstinputlisting{painter.scm}
\\
\section{square-limitに適用}
\lstset{numbers=left,basicstyle=\small}
\lstinputlisting{perorin-square-limit.scm}
画像はperorin.pngに保存してあります。\\
\\
\section{空間充填曲線を1種類作成}
\lstset{numbers=left,basicstyle=\small}
\lstinputlisting{hilbert.scm}
Hilbert circleを作成しました。\\
画像はhilbert.pngにあります。\\
\section{フラクタルを1種類作成}
\lstset{numbers=left,basicstyle=\small}
\lstinputlisting{snowflake.scm}
Kock Snowflakeを作成しました。\\
画像はsnowflake.pngにあります。\\

\end{document}
