\documentclass[a4paper,12pt]{article}
\usepackage{listings}
\title{アルゴリズムとデータ構造入門 第九回課題}
\author{1029-24-9540 山崎啓太郎}
\begin{document}
\maketitle

\section{複素数システム}
\lstset{numbers=left,basicstyle=\small}
\lstinputlisting{complex.scm}
\\
\section{実行例}
(stringify-complex (make-from-real-imag 3 4)) =$>$ "3+4i"\\
(stringify-complex (make-from-real-imag 3 0)) =$>$ "3"\\
(stringify-complex (make-from-real-imag 0 3)) =$>$ "3i"\\
(stringify-complex (make-from-real-imag 0 1)) =$>$ "i"\\
(stringify-complex (make-from-real-imag 0 -1)) =$>$ "-i"\\
(stringify-complex (make-from-real-imag 3 1)) =$>$ "3+i"\\
(stringify-complex (make-from-real-imag 3 -2)) =$>$ "3-2i"\\
(stringify-complex (make-from-real-imag 3 -1)) =$>$ "3-i"\\
(stringify-complex (add-complex (make-from-real-imag 3 -1) (make-from-real-imag 2 4))) =$>$ "5+3i"\\
(stringify-complex (sub-complex (make-from-real-imag 3 -1) (make-from-real-imag 2 4))) =$>$ "1-5i"\\
(stringify-complex (mul-complex (make-from-real-imag 3 -1) (make-from-real-imag 2 4))) =$>$ "10+10i"\\
(stringify-complex (div-complex (make-from-real-imag 3 -1) (make-from-real-imag 2 4))) =$>$ "0.1-0.7i"\\
(stringify-complex (make-from-real-imag (make-from-real-imag 3 -1) (make-from-real-imag 2 4))) =$>$ "-1+i"\\
\\
\section{説明}
make-from-real-imag関数を使って、x+iyとなる形の複素数を生成している。\\
add-complex、sub-complex、mul-complex、div-complex関数にて四則演算が可能である。\\
また、make-from-real-imag関数には複素数を渡すこともでき、その際には返り値が完結になるよう計算される。\\
簡略化は、make-from-real-imag関数で生成された複素数を文字列化するstringify-complex関数にて行なっている。\\
例として、簡略化されないと"0+1i"と表示されるものが、"i"と表示されるような処理がされている。(その他の例は実行例に有)\\

\end{document}
