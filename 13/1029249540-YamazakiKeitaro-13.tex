\documentclass[a4paper,12pt]{article}
\usepackage{listings}
\title{アルゴリズムとデータ構造入門 第十三回課題}
\author{1029-24-9540 山崎啓太郎}
\begin{document}
\maketitle

\section{組み込み数・有理数・複素数システムを統合した汎用算術システム}
\lstset{numbers=left,basicstyle=\small}
\lstinputlisting{calc.scm}
\\
\section{実行例}
(make-scheme-number 5) =$>$ 5\\
(make-rational 5 6) =$>$ (rational 5 . 6)\\
(make-real 5.6) =$>$ (real . 5.6)\\
(make-complex-from-real-imag 33 5) =$>$ (complex rectangular 33 . 5)\\
(make-complex-from-mag-ang 5 2) =$>$ (complex polar 5 . 2)\\
(raise (make-scheme-number 5)) =$>$ (rational 5 . 1)\\
(raise (make-rational 5 3)) =$>$ (real . 1.6666666666666667)\\
(raise (make-real 5.3)) =$>$ (complex rectangular 5.3 . 0)\\
(drop (make-complex-from-real-imag 33 0)) =$>$ (real . 33)\\
(drop (make-real 32)) =$>$ (rational 32 . 1)\\
(drop (make-rational 32 1)) =$>$ 32\\
(add (make-complex-from-real-imag 33 5) (make-rational 4 5)) =$>$ (complex rectangular 33.8 . 5)\\
(sub (make-complex-from-real-imag 33 5) (make-real 4.5)) =$>$ (complex rectangular 28.5 . 5)\\
(mul (make-scheme-number 5) (make-complex-from-mag-ang 5 2)) =$>$ (complex polar 25.0 . 2.0)\\
(div (make-rational 33 5) (make-real 4.5)) =$>$ (real . 1.4666666666666666)\\
(div (make-rational 33 1) (make-real 3)) =$>$ 11\\
(add (make-complex-from-real-imag 5 3) (make-complex-from-real-imag 5 -3)) =$>$ 10\\
(add (make-complex-from-real-imag 2 1) (make-complex-from-real-imag 2.5 -1)) =$>$ (real . 4.5)\\
\\
\section{説明}
make-scheme-number:整数を生成する関数\\
make-rational:分母と分子から有理数を生成する関数\\
make-real:実数を生成する関数\\
make-complex-from-real-imag:実部と虚部から複素数を生成する関数\\
make-complex-from-mag-ang:半径と角度から複素数を生成する関数\\
また、型階層はscheme-number-$>$rational-$>$real-$>$complexであり、raise関数によって型階層を上げることができる。\\
逆に、型階層を下げる場合はdrop関数を使う。\\
四則演算用の関数が用意されておりそれぞれ、add(和)、sub(差)、mul(積)、div(商)である。\\
また、内部でsimplification関数が呼ばれることにより、演算結果の簡略化が行われている。\\

\end{document}
