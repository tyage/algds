\documentclass[a4paper,12pt]{article}
\usepackage{listings}
\title{アルゴリズムとデータ構造入門 第四回課題}
\author{1029-24-9540 山崎啓太郎}
\begin{document}
\maketitle

\section{Ackermann関数}
\lstset{numbers=left,basicstyle=\small}
\lstinputlisting{ack.scm}

\section{出力結果}
(ack 0 2) =$>$ 3\\
(ack 1 2) =$>$ 4\\
(ack 0 2) =$>$ 7\\
(ack 0 2) =$>$ 29\\

\section{教科書練習問題Ex1.5}
{\large 解釈系が作用的順序の評価の時}\\
(test 0 (p))が評価される時、引数である0, (p)が評価される。\\
しかし、pはp自身を評価する関数であるため、無限ループし計算が終了しない。\\
\\
{\large 解釈系が正規順序の評価の時}\\
(test 0 (p))が評価される時、(if (= x 0) 0 y)が評価され、\\
xが0であることから0を返す。\\

\end{document}
