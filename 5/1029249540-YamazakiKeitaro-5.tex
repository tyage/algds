\documentclass[a4paper,12pt]{article}
\usepackage{listings}
\title{アルゴリズムとデータ構造入門 第五回課題}
\author{1029-24-9540 山崎啓太郎}
\begin{document}
\maketitle

\section{Section 1.29}
\lstset{numbers=left,basicstyle=\small}
\lstinputlisting{1.29.scm}
\\
\large{出力結果}\\
(simpson-integral (lambda (x) (* x x x)) 0 1 100) =$>$ 0.25\\
(simpson-integral (lambda (x) (* x x x)) 0 1 1000) =$>$ 0.25\\
(integral (lambda (x) (* x x x)) 0 1 1000) =$>$ 0.25050025000000004\\
(integral (lambda (x) (* x x x)) 0 1 100) =$>$ 0.25502500000000006\\
\\
\large{比較結果}\\
integralはnを増やすごとに0.25に近づいていることがわかる。\\
simpsonの公式を使った場合、分割数が少なくても正確な値が出る。\\
\\
\section{Section 1.31}
\lstset{numbers=left,basicstyle=\small}
\lstinputlisting{1.31.scm}
\\
\large{a.出力結果}\\
(pi 80) =$>$ 3.1513038442382775\\
\\
JAKLDではn=80までしか出力できなかったため、以下はgaucheで実行した。\\
\\
(display (exact-$>$inexact (pi 1000))) =$>$ 3.142377365093878\\
(display (exact-$>$inexact (pi 10000))) =$>$ 3.1416711865344635\\
\\
nが増えるごとにπに近づいていることがわかる。\\
\\
\large{b.再帰型、反復型の級数の積}\\
再帰型はproduct-recur、反復型はproduct-iterで定義してある。\\


\end{document}
